\chapter{Background or Project Scope}

\section{Cloud Computing Overview and Definition}

Cloud computing is commonly seen as a revolution in the world of distributed computing, offering easy online access to storage, applications and more as needed which makes it very attractive to individuals and enterprises alike. This concept of accessing computing resources without local IT infrastructure took the world by storm around the early 2000s, when "the cloud" started becoming a well-known term. The use of cloud services has been rapidly increasing ever since, with the worldwide cloud computing market size expected to be around \$500 billion in 2022 (GVR) and most IT decision-makers claiming that, in a few years' time, 95\% of all workloads will be carried out in the cloud.(logicmonitor)

So, what exactly is cloud computing? While no common conclusion has been reached among experts in the field (cloud many), the National Institute of Standards and Technology (NIST) at the U.S. Department of Commerce has proposed a comprehensive formal definition that seems to be in line with many people's perceptions regarding Cloud Computing.

NIST states that: \blockquote{ \textit{ Cloud computing is a model for enabling ubiquitous, convenient, on-demand network access to a shared pool of configurable computing resources (e.g., networks, servers, storage, applications, and services) that can be rapidly provisioned and released with minimal management effort or service provider interaction.}}

Five elements that characterize cloud computing are also mentioned (NIST,C2,CMANY):

\begin{enumerate} [I]
	\item \textit{On-demand self-service:} A customer can gain access to resources like storage and applications automatically when needed, without interacting with the cloud service provider. (CSP) 
	
	\item \textit{Broad network access:} Computer resources can be conveniently provisioned online via mechanisms that allow the use of (heterogeneous) client devices (e.g. laptops, smartphones etc.) 
	
	\item \textit{Resource Pooling:} CSP resources are gathered in a common pool using virtualization, allowing multiple customers to share a single resource (multi-tenancy) and dynamically assigning physical and virtual resources to match demand. Using such a pool-based model has many benefits like higher economies of scale, speed and availability but users do not control or are aware of where the provisioned resources are located. In some cases, the location at a high abstraction level (i.e. country, state etc.) can be specified. 
	
	\item \textit{Rapid Elasticity:} Access to cloud capabilities, which seem limitless to the consumer, does not require any up-front commitment as they can be elastically availed and released exactly when scaling up or down is deemed necessary according to demand. 
	
	\item \textit{Measured Service:} Despite the fact that customers share a common pool of resources, there are mechanisms in place to measure individual usage for each customer, hence providing transparency to both CSPs and consumers and allowing for resource control and optimization. Accordingly, cloud customers are commonly charged with a pay-per-use pricing model.
\end{enumerate}


\section{Service Models}
Cloud providers mainly offer three service models to consumers: (NIST,C2,C1):

\begin{description}
	\item [\textit{Software as a Service (SaaS)}] SaaS allows the customer to use application provided and hosted online by the CSP, which can be accessed using client devices via either a program interface or a thin interface like a web-browser. SaaS consumers do not have any control over the underlying cloud infrastructure, with the exception of some possible configuration changes that can be made to the application. Some characteristic examples of SaaS include SalesForce and Zoom.
	
	\item [\textit{Platform as a Service (PaaS)}] A customer using PaaS can access a development environment to build  and deploy applications using services, tools, programming languages and more provided by the CSP (or by another source). Once again, the customer does not control the cloud infrastructure but can control any applications they deployed and can possibly make configuration changes to the hosting environment. An example is Google's App Engine.
	
	\item [\textit{Infrastructure as a Service (IaaS)}] With IaaS, the customer can directly avail It resources such as networks, servers and storage. The customer has control over storage, OSs and deployed apps and possibly restricted control over host firewalls and other networking elements. Amazon's Simple Storage Service (S3) is an IaaS example.
\end{description}


\section{Deployment Models}
Three common types of cloud deployment models in total have been defined: (Nist, C2, Azure)

\begin{description}
	\item [\textit{Public cloud}] This type of cloud, which is the most popular choice, is openly available for any individual to access. The infrastructure is owned and managed exclusively by the cloud provider, and is located on their premises. Public clouds often involve lower costs but access to computing resources is shared with other users (tenants). An example of public cloud is Microsoft Azure.
	
	\item [\textit{Private cloud}] The resources offered by private clouds are exclusively dedicated to and used by a single organization using a private network and the infrastructure can be hosted on-site or on a third-party's premises. This model allows more flexibility when customizing the environment and is more secure, as there is no multi-tenancy.
	
	\item [\textit{Hybrid cloud}] Hybrid clouds involve a combination of different environments. The most common example is a combination of a public cloud and a private (on premises infrastructure) cloud. With hybrid clouds, companies can use on-premises infrastructure for workloads where privacy and low latency is essential, while also scaling up as needed by availing extra resources from the public cloud. 
\end{description}


\section{Challenges of Cloud Computing}
While the cloud can offer organizations a great many benefits such as elasticity, cost-efficiency and convenience for accessing computing resources, adopting the cloud also involves certain challenges that prospective customers need to be aware of: (C2, ambrust)

\begin{description}
	\item [\textit{Security}] One of the biggest deterring factors when it comes to cloud computing is the issue of security and confidentiality. After all, using the cloud typically involves trusting essential data to an unknown third party and becoming exposed to both external and internal threats. External dangers are no different than those that any large data center faces but, in the cloud, security responsibilities are distributed among different parties (users responsible with application security, while CSPs responsible with physical security etc.) In addition, cloud consumers face internal threats, namely the other users. Any error or security hole during the virtualization process might allow one user to access sensitive data of another and reputation fate sharing is also a possibility if resources are shared with someone with a criminal mind. Finally users often rely on contracts to protect themselves against provider malfeasance but accidents can still occur (e.g data permissions bug rendering sensitive information visible)
	
\end{description}

.....(section not complete)



\section{Summary}
Summarise what was said in this chapter and link with the next chapter.