\chapter{Conclusion and Future Work}\label{ch:conclusion}

\section{Conclusion}
In this paper, we provided a brief overview of cloud technologies in general, and then focused more on object and file storage services, defining them and comparing the performance, security, pricing schemes and features of some well-known services (Dropbox, Box and Google Drive for file storage, and Amazon S3 and Azure Blob Storage for object storage). Aside from the obvious differences between these two storage types, with object storage being ideal for huge amounts of unstructured data thanks to the scalability of its flat namespace, and file storage facilitating the sharing and real-time collaboration of files, there also exist many differing characteristics among services of the same storage type. To illustrate, while Blob Storage's SLA guarantees better durability, Amazon S3's promises fast replication speed, while allowing redundancy settings to be specified at the object-level as opposed to  Blob Storage's account-level redundancy settings. Similarly, for the file storage services, Dropbox provides the best collaboration tools with Dropbox Paper, while only Box allows users to specify where data should be stored with Box Zone, and only Google Drive supports client-side encryption.Therefore, by highlighting such differences, we emphasize the importance that prospective users research all potential services before choosing which ones best suit their needs. This comparison also helped make clear why users might make use of many such services at ones, which our library aims to facilitate by providing a sine interface for  many cloud storage services. Then, we covered the design of our python library which, thus far, allows the use of Dropbox, Box, Google Drive and Amazon S3 for uploading, downloading and deleting files. We also outlined our implementation of the library, highlighting the different capabilities (regarding authentication and file management) of each SDK that we used in the process, and any resulting challenges we had to overcome. Finally, we evaluated the performance of our library by measuring the execution time when uploading, downloading or deleting files of various sizes. From this evaluation, only Google Drive severely under-performed when uploading large files (likely due to speed limitations set by the cloud provider). However, smaller differences were also observed as, for instance, Dropbox is the fastest when uploading small files, while S3 performs the best for large file uploads. In general, though, performance among all services is consistent and comparable, making our library a valid option for using these services. 

\section{Future work}
There exist various possible avenues for improving and building on our library in the future. Specifically, we believe it is worth investigating the use of multithreading when managing multiple files, to potentially improve performance.  Furthermore, the implementation of a progress bar when uploading or downloading large files could especially prove useful by keeping users informed of the progress at any point. The functionality offered by the library  could also be enhanced to provide more than the upload, download and deletion of files. For instance, providing the ability to move or rename files, to modify tags/labels of files or to control access to files (e.g. via file locks or S3's \ac{acl}) are all possible with the CSP-provided SDKs, and could improve the flexibility and usability of our tool. Finally, our library provides a solid basis for implementing additional cloud storage services other than the current four, to more reliably cover all the storage needs of users. 