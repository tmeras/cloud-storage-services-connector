\tableofcontents

\chapter{Introduction}

\section{Introduction}
Cloud technologies are becoming more popular by the day, as greater and greater amounts of people takes advantage of them to gain access to a seemingly infinite and highly configurable array of computing resources via the internet, on a pay-as-you-go basis ~\cite{gvr}. Using the cloud, people can remotely access scalable infrastructure like storage, servers and networks systems, all of which is managed and maintained by the \ac{csp}, to then host web apps, conduct high-performance computing or even to migrate the entirety of an organization's on-site IT infrastructure to the cloud~\cite{iaas}. \ac{csp}s can also provide environments which support and simplify the entire development lifecycle of cloud-native applications, regardless of complexity. Furthermore, customers frequently use all kinds of applications which are hosted on the cloud, like email of video conferencing applications.

For the purposes of this project, we will focus on cloud storage, which allows data or files to be stored on the internet in a highly scalable manner, all while being accessible at any time and from anywhere~\cite{s3_cloud_storage}. Cloud storage has a variety of use cases, like performing analytics on vast amounts of data, setting up data lakes, providing data management capabilities to applications or replicating data for backup and recovery, all of which can be accomplished thanks to the excellent scalability of the flat data environment used by object storage. At the same time, file storage services are extensively used by users to synchronize files across various devices and easily share them with others, hence facilitating collaboration~\cite{objectvsfile,objectvsfile2}. However, each cloud service has its own interface for communicating with the storage servers which can easily become quite inconvenient for users that wish to make use many such services at once. Therefore, the main goal of this project is to remedy this situation by implementing a single interface that allows for interaction with a number of cloud storage services.

\section{Aims and Objectives}
Mention what you aim to do for project. For every aim you have you must also mention what are the objectives that you must achieve in order to fulfil the aim.

\section{Structure of the thesis}
Explain to the reader what is included in the remainder of the thesis.

\section{Summary}
Summarise what was said in this chapter and link with the next chapter.

