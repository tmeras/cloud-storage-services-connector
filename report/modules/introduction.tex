\chapter{Introduction}

\section{Introduction}
The popularity of cloud technologies is growing by the day, with more and more people taking advantage of them to access a seemingly infinite and highly configurable array of computing resources via the internet, on a pay-as-you-go basis ~\cite{gvr}. Using the cloud, people can remotely access scalable infrastructure like storage, servers and networks systems, all of which is managed and maintained by the \ac{csp}, to then host web apps, conduct high-performance computing or even to migrate the entirety of an organization's on-site IT infrastructure to the cloud~\cite{iaas}. \ac{csp}s can also provide environments which support and simplify the entire development lifecycle of cloud-native applications, regardless of complexity~\cite{paas}. Furthermore, customers frequently use all kinds of applications which are hosted on the cloud, like email of video conferencing applications~\cite{saas}.

For the purposes of this project, we focus on cloud storage, which allows data or files to be stored on the internet in a highly scalable manner, all while being accessible at any time and from anywhere~\cite{s3_cloud_storage}. Cloud storage has a variety of use cases, like performing analytics on vast amounts of data, setting up data lakes, providing data management capabilities to applications or replicating data for backup and recovery, all of which can be accomplished thanks to the excellent scalability of the flat data environment used by object storage. At the same time, file storage services are extensively used by users to synchronize files across various devices and easily share them with others, while setting access permissions at the user level, hence facilitating collaboration~\cite{objectvsfile,objectvsfile2}. However, each cloud service has its own interface for communicating with the storage servers which can easily become quite inconvenient for users that wish to make use many such services at once. Therefore, the main goal of this project is to remedy this situation by implementing a single interface that allows for interaction with a number of cloud storage services.

\section{Aims and Objectives}
The goals of this project are:

\begin{enumerate}
    \item Research and compare different cloud-based storage service providers, highlighting their differences in many areas including pricing, performance, security and capabilities. The services that we compare are: Amazon S3 and Azure Blob Storage for object storage, and Dropbox, Box and Google Drive for file storage.

    \item Study the Python Software Development Kits (SDKs) offered by various popular cloud storage services, namely Dropbox, Box, Google Drive and Amazon S3 to understand their authentication and file manipulation mechanisms, deriving a common interface across all of these.

    \item Implement a Python library that provides this common interface, hence allowing authentication with the selected services as well as uploading, downloading and deleting files or folders to/from all them, while hiding the implementation details of each service.

    \item Evaluate the performance of our library by measuring the execution time required for the upload, download and deletion of files.
\end{enumerate}

\section{Structure of the thesis}
The remainder of the thesis is structured as follows. In chapter 2 we provide some background information on cloud technologies by outlining their essential characteristics, the different cloud service and deployment models, as well as the various challenges involved with using cloud computing. In chapter 3, we define cloud object storage and file storage services along with their use cases, and compare some popular examples of each service type in terms of their performance, security, pricing and capabilities. Chapter 4 goes over the design of our library by presenting and explaining our design model, and briefly covers OAuth, which is a authorization protocol commonly used by many cloud services. In chapter 5 we cover the implementation details of our library, specifying how we implemented user authentication and upload/download/deletion of files using the official SDKs of each selected service, along with any particularities or difficulties we faced. In chapter 6, we present and discuss the results obtained from the evaluation our library, which was conducted by uploading, downloading and deleting files (between 1-1000MB in size) to and from each of the services supported by the library. Finally, concluding remarks are given and potential future work is described in chapter 7.
